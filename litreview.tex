\documentclass[12pt,a4paper]{article}
\usepackage{lipsum}
\usepackage[T1]{fontenc}
\usepackage[utf8]{inputenc}
\usepackage[noadjust]{cite}
\usepackage{authblk}
\usepackage[top=2cm, bottom=2cm, left=2cm, right=2cm]{geometry}
\usepackage{fancyhdr}
\usepackage{listings}
\usepackage{csquotes}
\usepackage{color}

\definecolor{gray}{rgb}{0.5,0.5,0.5}
\definecolor{lightgray}{rgb}{0.9,0.9,0.9}
\definecolor{editorGray}{rgb}{0.95, 0.95, 0.95}
\definecolor{editorMauve}{rgb}{0.58,0,0.82}
\definecolor{editorGreen}{rgb}{0, 0.5, 0} % #007C00 -> rgb(0, 124, 0)
\definecolor{editorBrown}{rgb}{.75, 0.375, 0} % #FF7F00 -> rgb(239, 169, 0)

\lstdefinelanguage{JavaScript}{
	morekeywords={typeof, new, true, false, catch, function, return, null, catch, switch, var, if, in, while, do, else, case, break},
	morecomment=[s]{/*}{*/},
	morecomment=[l]//,
	morestring=[b]",
	morestring=[b]'
}

\lstdefinelanguage{HTML5}{
	language=html,
	sensitive=true, 
	alsoletter={<>=-},
	otherkeywords={
		% HTML tags
		<html>, <head>, <title>, </title>, <meta, />, </head>, <body>,
		<canvas, \/canvas>, <script>, </script>, </body>, </html>, <!, html>, <style>, </style>, ><
	},  
	ndkeywords={
		% General
		=,
		% HTML attributes
		charset=, id=, width=, height=,
		% CSS properties
		border:, transform:, -moz-transform:, transition-duration:, transition-property:, transition-timing-function:
	},  
	morecomment=[s]{<!--}{-->},
	tag=[s]
}

\lstset{%
	% Basic design
	backgroundcolor=\color{editorGray},
	basicstyle={\small\ttfamily},   
	frame=l,
	% Line numbers
	xleftmargin={0.75cm},
	numbers=left,
	stepnumber=1,
	firstnumber=1,
	numberfirstline=true,
	% Code design   
	keywordstyle=\color{blue}\bfseries,
	commentstyle=\color{editorGreen}\ttfamily,
	ndkeywordstyle=\color{editorBrown}\bfseries,
	stringstyle=\color{editorMauve},
	% Code
	language=HTML5,
	alsolanguage=JavaScript,
	alsodigit={.:;},
	tabsize=2,
	showtabs=false,
	showspaces=false,
	showstringspaces=false,
	extendedchars=true,
	breaklines=true,        
	% Support for German umlauts
	literate=%
	{Ö}{{\"O}}1
	{Ä}{{\"A}}1
	{Ü}{{\"U}}1
	{ß}{{\ss}}1
	{ü}{{\"u}}1
	{ä}{{\"a}}1
	{ö}{{\"o}}1
}

%
\pagestyle{fancy}
%
\renewenvironment{abstract}{%
	\hfill\begin{minipage}{0.95\textwidth}
		\rule{\textwidth}{1pt}}
	{\par\noindent\rule{\textwidth}{1pt}\end{minipage}}
%
\makeatletter
\renewcommand\@maketitle{%
	\hfill
	\begin{minipage}{0.95\textwidth}
		\vskip 2em
		\let\footnote\thanks 
		{\LARGE \@title \par }
		\vskip 1.5em
		{\large \@author \par}
	\end{minipage}
	\vskip 1em \par
}
\makeatother


%
\begin{document}
	%
	%title and author details
	\title{Title, i.e. "CS 252 Project Report: Faceted Values in Haskell", make sure it fits on one line}
	\author[1]{Andrew Kalenda\thanks{andrew.kalenda@gmail.com}}
	\affil[1]{Department of Computer Science, San Jos\'{e} State University}
	%
	\maketitle
	%
	\begin{abstract}
	The abstract goes here. This is a template that I put together as a starting point for writing any sort of Computer Science research paper. A lot of the notes I have in various sections have been taken without attribution (sorry!) from various corners of the Internet. Suggestions on how to beautify papers are appreciated!
	
	\end{abstract}
	
	\section{Introduction}
	
	Regarding the abstract, it should be a summary of the paper, considering the following:
	 
	 \textbf{Motivation}:
	 Why do we care about the problem and the results? If the problem isn't obviously "interesting" it might be better to put motivation first; but if your work is incremental progress on a problem that is widely recognized as important, then it is probably better to put the problem statement first to indicate which piece of the larger problem you are breaking off to work on. This section should include the importance of your work, the difficulty of the area, and the impact it might have if successful.
	 
	 \textbf{Problem statement}:
	 What problem are you trying to solve? What is the scope of your work (a generalized approach, or for a specific situation)? Be careful not to use too much jargon. In some cases it is appropriate to put the problem statement before the motivation, but usually this only works if most readers already understand why the problem is important.
	 
	 \textbf{Approach}:
	 How did you go about solving or making progress on the problem? Did you use simulation, analytic models, prototype construction, or analysis of field data for an actual product? What was the extent of your work (did you look at one application program or a hundred programs in twenty different programming languages?) What important variables did you control, ignore, or measure?
	 
	 \textbf{Results}:
	 What's the answer? Specifically, most good computer architecture papers conclude that something is so many percent faster, cheaper, smaller, or otherwise better than something else. Put the result there, in numbers. Avoid vague, hand-waving results such as "very", "small", or "significant." If you must be vague, you are only given license to do so when you can talk about orders-of-magnitude improvement. There is a tension here in that you should not provide numbers that can be easily misinterpreted, but on the other hand you don't have room for all the caveats.
	 
	 \textbf{Conclusions}:
	 What are the implications of your answer? Is it going to change the world (unlikely), be a significant "win", be a nice hack, or simply serve as a road sign indicating that this path is a waste of time (all of the previous results are useful). Are your results general, potentially generalizable, or specific to a particular case?
	
	\begin{itemize}
		\item Abstracts are important when others are scanning through large databases of scholarly works. Make it as succinct and useful as possible, a single easily readable paragraph.
		\item Meet the word count limitation. If your abstract runs too long, either it will be rejected or someone will take a chainsaw to it to get it down to size. Your purposes will be better served by doing the difficult task of cutting yourself, rather than leaving it to someone else who might be more interested in meeting size restrictions than in representing your efforts in the best possible manner. An abstract word limit of 150 to 200 words is common.
		\item Any major restrictions or limitations on the results should be stated, if only by using "weasel-words" such as "might", "could", "may", and "seem".
		\item Think of a half-dozen search phrases and keywords that people looking for your work might use. Be sure that those exact phrases appear in your abstract, so that they will turn up at the top of a search result listing.
		\item Usually the context of a paper is set by the publication it appears in (for example, IEEE Computer magazine's articles are generally about computer technology). But, if your paper appears in a somewhat un-traditional venue, be sure to include in the problem statement the domain or topic area that it is really applicable to.
		\item Some publications request "keywords". These have two purposes. They are used to facilitate keyword index searches, which are greatly reduced in importance now that on-line abstract text searching is commonly used. However, they are also used to assign papers to review committees or editors, which can be extremely important to your fate. So make sure that the keywords you pick make assigning your paper to a review category obvious (for example, if there is a list of conference topics, use your chosen topic area as one of the keyword tuples).
	\end{itemize}
	
	The introduction is much like the abstract, but does not include findings or conclusions. It does include more details on the context of the problem, questions that are to be answered by the paper, and a working hypothesis where appropriate. Even if this isn't a lab report, you can still often make a more complete and organized paper by constructing it like one!
	
	\section{Section name}
	\textit{Italics.} \textbf{Boldface}. Normal text. Normal text with citation\cite{HarryPotter}. Code listing:
	
	\lstset{language=JavaScript}
	\begin{lstlisting}
	// Heyo, comments
	var b = user.getBalance()
	var isRich = b > 999999
	if (isRich)
	  out = 'Youre a millionaire!'
	else
	  out = 'Youre not a millionaire!'
	save(out)
	print(out)
	\end{lstlisting}
	
	\begin{lstlisting}
		<!DOCTYPE html>
		<html>
		  <head>
		    <title>Canvas</title>
		    <meta charset="UTF-8" />
		    <style>
		      #square {
		        border: 1px solid black;
		                transform: scale(10) rotate(3deg) translateX(0px);
		                -moz-transform: scale(10) rotate(3deg) translateX(0px);
		      }
		
		            .box {              
		                transition-duration: 2s;
		                transition-property: transform;
		                transition-timing-function: linear;
		      }
		    </style>
		  </head>
		  <body>
		    <canvas id="square" width="200" height="200"></canvas>
		    <script>
		            var canvas = document.createElement('canvas');
		            canvas.width = 200;
		            canvas.height = 200;
		
		            var image = new Image();
		            image.src = 'images/card.png';
		            image.width = 114;
		            image.height = 158;
		            image.onload = window.setInterval(function() {
		                rotation();
		            }, 1000/60);
		   </script>
		  </body>
		</html>
	\end{lstlisting}
	
	Numbered list, it looks kinda ugly to me though:
	
	\begin{enumerate}
		\item bleagh
		\item blargh
		\item blooph
	\end{enumerate}
	
	Text before block quote:
	
	\blockcquote{HarryPotter}{“The Dark Arts [...] are many, varied, ever-changing, and eternal. Fighting them is like fighting a many-headed monster, which, each time a neck is severed, sprouts a head even fiercer and cleverer than before. You are fighting that which is unfixed, mutating, indestructible. [...] Your defenses [...] must therefore be as flexible and inventive as the arts you seek to undo."}
	
	Text after block quote. Some inline $ math^{a}_{b}=\sqrt{\frac{c}{d}} $. Some block math:
	
	$$ math^{a}_{b}=\sqrt{\frac{c}{d}} $$

	And so on.
	
	\section{Results}
	If this is a report on something experimental, sort hard data into this section

	\section{Conclusion}
	Restate the introduction, but this time addressed to someone who has the knowledge your paper offers:
	\begin{itemize}
		\item What you did
		\item What you discovered
		\item If you operated from a hypothesis, was it supported?
		\item Link hard data results to hypothesis
		\item What was learned?
		\item Answer any questions explicitly raised in the introduction
		\item Were objectives achieved?
		\item Errors and problems encountered?
		\item Any remaining uncertainties?
		\item Propose avenues for future exploration, experimentation
		\item Propose additional questions
		\item Relate your research to others'
		\item Final statement
	\end{itemize}
	
	\begin{thebibliography}{9}
		
		\bibitem{HarryPotter}
		Rowling, J. K. (2013). 
		\textit{Harry Potter and the Half-Blood Prince (Vol. 6). }
		Bloomsbury Publishing.
		
	\end{thebibliography}
	
\end{document}

